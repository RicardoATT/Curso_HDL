\section{Conclusiones}
En conclusión, se implementó la simulación del programa de asignación de valores en la salida del procesador NIOS II de manera exitosa.

Utilizando el procedimiento de la presentación vista en clase para simular un NIOS II, se escribieron los comandos necesarios para simular la impresión de una cadena de texto en la consola de ModelSim y en el visor de formas de onda se observó el tiempo requerido para cargar el mensaje y para mandar valores al puerto de salida del procesador. Se empleó la herramienta de \textit{Eclipse} para variar el retardo del programa y con la herramienta de ModelSim se generó el archivo \textit{.do} para los comandos.

En los Anexos se pueden encontrar los códigos implementados.