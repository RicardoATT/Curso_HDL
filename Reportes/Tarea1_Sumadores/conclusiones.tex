\section{Conclusiones}
En conclusión, se implementaron los 3 circuitos en ambos lenguajes de manera exitosa.

Para el sumador con acarreo de entrada, se implementó la descripción por comportamiento, usando el operador aritmético ``+'' y con el visor RTL se entendió como es que el lenguaje interpreta la suma de más de dos variables de entrada.

Para el sumador completo, se comprendió como es que se debe manipular la concatenación para realizar una correcta descripción de operaciones (en este caso, la suma con acarreo de salida) y evitar errores en el tamaño de las variables utilizadas.

Para el sumador/restador, se utilizaron las sentencias \textit{if-else-then} para generar un multiplexor, el cual selecciona que resultado visualizar en la salida, con una variable de control evaluada en las sentencias mencionadas.

En los Anexos se pueden encontrar los códigos implementados junto con sus respectivos bancos de pruebas.