\section{Conclusiones}
En conclusión, se implementaron los 4 circuitos en lenguaje Verilog de manera exitosa.

Para el desplazador lógico, se implementó utilizando concatenaciones en lugar de utilizar los operadores específicos ``<<'' y ``>>'', por lo que se entendió a profundidad el funcionamiento de este modulo. 

Para el multiplexor, se implementó su descripción de dos formas, observando que con la estructura \textit{case} se generó un multiplexor, mientras que con la estructura \textit{if-else} se generaron otros elementos lógicos, como comparadores y multiplexores de menor tamaño.

Para la memoria ROM, se observó que este modulo se implementa de manera similar a un decodificador, diferenciándose en el tamaño, ya que la memoria puede tener cualquier tamaño.

Para el buffer tri-estado, con la herramienta de \textit{Chip Planner}, se entendió como es que el entorno asignó el modulo al bloque de entradas y salidas, ya que en ese lugar se utilizan buffers para intercambiar el funcionamiento de los puertos entre entrada y salida.

En general, comprendió como es que operan dichos circuitos combinatorios y como es que se implementan en el visor RTL. Se comprobó su funcionamiento utilizando las simulaciones de forma de onda en ModelSim y se asignaron los pines correspondientes en la placa de desarrollo para programar el dispositivo y realizar las pruebas pertinentes en hardware.

En los Anexos se pueden encontrar los códigos implementados junto con sus respectivos bancos de pruebas.