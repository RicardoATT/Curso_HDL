\section{Conclusiones}
En conclusión, se implementaron los 4 circuitos en lenguaje Verilog de manera exitosa.

Para los flip flop tipo D, se implementaron de manera correcta y se diferenció el funcionamiento del \textit{reset} asíncrono del síncrono, así como sus características físicas en el visor RTL. 

Para los contadores de 8 bits, se describieron de forma adecuada. Para el contador sencillo se observo como se implementa en el visor RTL, utilizando unicamente un flip flop de 8 bits junto con un sumador en el lazo de retroalimentación. Para el contador completo, se entendió como implementar la señales de control con respecto a un orden de prioridad y se observó la complejidad de este modulo al implementar no solo un flip flop de 8 bits, sino también sumadores, multiplexores y comparadores.

Se comprobó su funcionamiento utilizando las simulaciones de forma de onda en ModelSim y se asignaron los pines correspondientes en la placa de desarrollo para programar el dispositivo y realizar las pruebas pertinentes en hardware.

En los Anexos se pueden encontrar los códigos implementados junto con sus respectivos bancos de pruebas.