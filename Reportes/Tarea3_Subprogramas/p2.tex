\section{Funciones y tareas en Verilog \label{sec:s2}}

\begin{center}
	\begin{minipage}{12cm}
		\begin{tcolorbox}[title=Actividad 2]
			En verilog existen las funciones (functions) y las tareas (tasks) como subprogramas. Hacer una breve descripción de sus características (como la que aparece al inicio de la presentación y del documento para subprogramas).
		\end{tcolorbox}	
	\end{minipage}
\end{center}

En Verilog, una función o tarea es un grupo de declaraciones que realiza alguna acción específica. Se puede llamar a ambos en varios puntos para realizar una determinada operación. También se utilizan para dividir código grande en partes más pequeñas para facilitar su lectura y depuración. \cite{vlsiverify}

\subsection{Definiciones}

Las tareas se utilizan en todos los lenguajes de programación, generalmente conocidas como procedimientos o subrutinas. Los datos se pasan a la tarea, se realiza el procesamiento y se devuelve el resultado. Tienen que ser llamados específicamente, con datos de entrada y salidas. Incluidos en el cuerpo principal del código, se pueden llamar muchas veces, lo que reduce la repetición del código.
	
Ahora bien, una función es lo mismo que una tarea, con muy pequeñas diferencias, como que la función no puede controlar más de una salida y no puede contener retrasos. 

\subsection{Sintaxis}

La sintaxis de una tarea se observa en el \autoref{prog:verilog_task_syntax}.

\begin{itemize}
	\item Una tarea comienza con la palabra clave \textit{task} y termina con la palabra clave \textit{endtask}.
	\item Las entradas y salidas se declaran después de la palabra clave \textit{task}.
	\item Las variables locales se declaran después de la declaración de entradas y salidas. \cite{asic_2014} 
\end{itemize}

\lstinputlisting[
language=verilog,
label=prog:verilog_task_syntax,
caption={Sintaxis de la descripción de una tarea en Verilog. \cite{vlsiverify}}
]{
	codes/Verilog_Task_Syntax.v
}

La sintaxis de una función se observa en el \autoref{prog:verilog_function_syntax}.

\begin{itemize}
	\item Una función comienza con la palabra clave \textit{function} y termina con la palabra clave \textit{endfunction}.
	\item Las entradas se declaran después de la palabra clave \textit{endfunction}. \cite{asic_2014} 
\end{itemize}

\lstinputlisting[
language=verilog,
label=prog:verilog_function_syntax,
caption={Sintaxis de la descripción de una función en Verilog. \cite{vlsiverify}}
]{
	codes/Verilog_Function_Syntax.v
}

\subsection{Comparación}

\begin{itemize}
	\item La tarea puede tener cero o más de un argumento. La función debe tener al menos un argumento. 
	\item Una tarea puede tener declaraciones de retraso en su interior. Una función no puede tener una declaración de retraso. Debería devolver un valor en el mismo paso de tiempo.
	\item Una tarea no tiene un tipo de retorno. Sin embargo, los argumentos de salida ayudan a devolver valores. Una función tiene un tipo de retorno.
	\item Una tarea puede devolver más de un valor, ya que puede haber cualquier número de argumentos de salida. Una función puede devolver solo un valor ya que los argumentos de salida no se pueden usar en funciones.
	\item Una tarea puede llamar a otra función o tarea desde su cuerpo. Una función sólo puede llamar a otra función desde su cuerpo. No se puede llamar a una tarea porque puede consumir tiempo y no se permite que la función consuma tiempo.
\end{itemize}

\subsection{Variantes}

Un dato interesante es que las tareas y funciones son de 2 tipos, estáticas y automáticas.

De forma predeterminada, todas las funciones y tareas son estáticas en Verilog. Las tareas estáticas significan que las variables declaradas dentro de una tarea conservarán su valor anterior cada vez que se llame a la tarea. Por tanto, se puede decir que la memoria asignada para la función o tarea permanece igual durante cada llamada, volviéndola estática.

Mientras que en la función o tarea automática, se asigna una nueva ubicación de memoria cada vez que se llaman. Por lo tanto, cualquier variable interna presente dentro de la tarea no conservará su valor anterior. \cite{octet_2021}