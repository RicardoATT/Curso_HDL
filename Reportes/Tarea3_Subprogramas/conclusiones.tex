\section{Conclusiones}
En conclusión, se implementaron los circuitos en VHDL y en Verilog de forma correcta.

Se comprendió y se diferenció a los procedimientos y a las funciones en VHDL por medio del análisis de la sintaxis de cada uno. De igual forma, se hizo lo mismo con las tareas y funciones en Verilog.

Se investigó de manera concisa la definición, las sintaxis, las similitudes y las diferencias entre las funciones y las tareas en Verilog.

Se implementó la descripción de 3 sumadores de 4 bits empleando los subprogramas de cada lenguaje y se observó con el visor RTL que el modulo es igual en todos los casos y por medio de las simulaciones de forma de onda en ModelSim se visualizó que la operación es la misma.

En los Anexos se pueden encontrar los códigos implementados junto con sus respectivos bancos de pruebas. Cabe señalar que los \textit{Test benches} no variaron mucho entre una implementación y otra.