\input{../Preambulo.tex}
\newcommand{\pnormal}[3]{
	\backgroundsetup{
		scale=1,
		opacity= 0.05,
		angle= 0,
		contents= {
			\includegraphics[width=0.75\paperwidth, height=0.9\paperheight]{../ImgPortada/IPN_Logo.png}
		}
	}
	\BgThispage
	\begin{center}
   		\begin{figure}[!tbp]
   			\centering
   			\subfloat{\includegraphics[scale=0.2]{../ImgPortada/CIC_Logo.png}}
   			\hfill
   			\subfloat{\includegraphics[scale=0.2]{../ImgPortada/MICROSE_Logo.png}}
   		\end{figure}
        
        {\LARGE \textcolor{guinda}{\textbf{I}nstituto \textbf{P}olit\'ecnico \textbf{N}acional} }
        \\ \vspace{1cm}
        {\LARGE \textit{ \textcolor{azul}{Centro de Investigaci\'on en Computaci\'on} } }
		\\
        {\Large \textit{ \textcolor{azul}{(\textbf{CIC})} } }
        \\ \vspace{2.5cm}
        {\Large \textsc{\underline{#1}}} %%%%%%%%%%Materia%%%%%%%%%%
        \\ \vspace{3cm}
        {\LARGE \textbf{ \textcolor{guinda}{#2} } } %%%%%%%%%%Título%%%%%%%%%%
        \\ \vspace{3cm}
        {\Large \textsc{ #3 } } %%%%%%%%%%Profesor%%%%%%%%%%
        \\ \vspace{2cm}
        {\Large \textsc{ Ing. Ricardo Aldair Tirado Torres } }
        %\ \\
        %A230720
        \\ \vspace{2cm}
        {\large \textsc{  \textsl{ Ciudad de M\'exico, \today } } }
    \end{center}
}




\begin{document}
	\pagenumbering{gobble}
	\pnormal
	{Lenguajes de descripción de hardware}
	{Práctica 5. Máquina de estados finitos.}
	{M. en C. Osvaldo Espinoza Sosa}
	\tableofcontents
	
	\pagenumbering{arabic}
	\newpage \section{Manejo de una memoria de 32 localidades de 8 bits \label{sec:s1}}

\begin{center}
	\begin{minipage}{12cm}
		\begin{tcolorbox}[title=Actividad 1]
			Crear una memoria inicializada de 32x8 con archivo MIF como se vio en clase, habilitar la opción de uso del \textit{In-System Memory Content Editor}. Automatizar el despliegue del contenido de las 32 localidades de memoria, usando un contador a una razón de localidad por segundo. Usar la tarjeta DE2-115.
		\end{tcolorbox}	
	\end{minipage}
\end{center}

La visualización RTL de la memoria de 32 localidades de 8 bits se muestra en la \autoref{fig:mem_rtl}. De manera interna, la implementación en hardware utiliza una instancia de memoria perteneciente a la herramienta de \textit{IP Catalog} (ver \autoref{fig:mem_rtl2}). Para inicializar la memoria, se generó un archivo \textit{.mif} cuyos valores se presentan en la \autoref{fig:mem_init}, siendo estos el valor de la dirección incrementado en uno. Las simulaciones se visualizan en la \autoref{fig:mem_wave}. Se observa que por cada ciclo de reloj se incrementa el valor de la dirección en uno y el valor almacenado en la localidad actual se ve reflejado en la salida Q. Una observación importante es que la señal WE esta en bajo, por lo que la memoria esta en modo lectura. Con respecto a la implementación en la tarjeta DE2-115, se utilizó la herramienta \textit{In-System Memory Content Editor}, como se visualiza en la \autoref{fig:mem_content1}

En los Anexos se localiza la descripción de la memoria de 32 localidades de 8 bits. Al emplear un elemento proveniente de la herramienta de \textit{IP Catalog}, el código se reduce a describir las entradas y salidas y llamar a la instancia dentro del módulo principal.

\begin{figure}[ht]
	\centering
	\includegraphics[scale=0.35]{Mem_RTL.png}
	\caption{Diagrama RTL de la memoria de 32 localidades de 8 bits, instanciada con un componente de IP Catalog. \label{fig:mem_rtl}}
\end{figure}

\begin{figure}[ht]
	\centering
	\includegraphics[scale=0.87]{Mem_RTL2.png}
	\caption{Diagrama RTL de la memoria de 32 localidades de 8 bits, instanciada con un componente de IP Catalog (vista interna). \label{fig:mem_rtl2}}
\end{figure}

\begin{figure}[ht]
	\centering
	\includegraphics[scale=1]{Mem_Init.png}
	\caption{Visualización del archivo de inicialización de la memoria de 32 localidades de 8 bits. \label{fig:mem_init}}
\end{figure}

\begin{figure}[ht]
	\centering
	\includegraphics[scale=0.75]{Mem_Wave.png}
	\caption{Simulación de la memoria de 32 localidades de 8 bits, en el visor de formas de onda de ModelSim. \label{fig:mem_wave}}
\end{figure}

\begin{figure}[ht]
	\centering
	\includegraphics[scale=0.75]{Mem_ModelSim.png}
	\caption{Simulación de la memoria de 32 localidades de 8 bits, en el visor de formas de onda de ModelSim. \label{fig:mem_modelsim}}
\end{figure}

\begin{figure}[ht]
	\centering
	\includegraphics[scale=0.75]{Mem_Content.png}
	\caption{Simulación de la memoria de 32 localidades de 8 bits, en el visor de formas de onda de ModelSim. \label{fig:mem_content1}}
\end{figure}

\begin{figure}[ht]
	\centering
	\includegraphics[scale=0.75]{Mem_Content2.png}
	\caption{Simulación de la memoria de 32 localidades de 8 bits, en el visor de formas de onda de ModelSim. \label{fig:mem_content2}}
\end{figure}

\begin{figure}[ht]
	\centering
	\includegraphics[scale=0.75]{Mem_Content3.png}
	\caption{Simulación de la memoria de 32 localidades de 8 bits, en el visor de formas de onda de ModelSim. \label{fig:mem_content3}}
\end{figure}

\begin{figure}[ht]
	\centering
	\includegraphics[scale=0.75]{Mem_Content4.png}
	\caption{Simulación de la memoria de 32 localidades de 8 bits, en el visor de formas de onda de ModelSim. \label{fig:mem_content4}}
\end{figure}
	\newpage \section{Comparador de magnitud \label{sec:s2}}

\begin{center}
	\begin{minipage}{12cm}
		\begin{tcolorbox}[title=Actividad 2]
			Completar el código del comparador de magnitud en el lenguaje de su elección, para entradas de 4 bits. Compilar y simular. Configurar en la tarjeta DE2-115, asignar interruptores como entradas y un LED para observar la salida.
		\end{tcolorbox}	
	\end{minipage}
\end{center}

La visualización RTL del comparador de magnitud en Verilog se muestra en la \autoref{fig:magnitude_comparator_rtl}. Como se observa, la implementación del comparador de magnitud se hace utilizando un comparador de igualdad que pone la salida en alto unicamente cuando ambas entradas tienen el mismo valor. Las simulaciones para el código en Verilog se visualizan en la \autoref{fig:magnitude_comparator_WaveBi} en base binaria y en la \autoref{fig:magnitude_comparator_WaveDe} en base decimal. Se utilizaron todos los valores posibles en las entradas para observar un comportamiento completo en la salida.

\begin{figure}[ht]
	\centering
	\includegraphics[scale=0.5]{Magnitude_Comparator_RTL.png}
	\caption{Diagrama RTL del comparador de magnitud. \label{fig:magnitude_comparator_rtl}}
\end{figure}

\begin{figure}[ht]
	\centering
	\includegraphics[scale=0.75]{Magnitude_Comparator_WaveBi.png}
	\caption{Simulación del comparador de magnitud con el visor de formas de onda de ModelSim (Base binaria). \label{fig:magnitude_comparator_WaveBi}}
\end{figure}

\begin{figure}[ht]
	\centering
	\includegraphics[scale=1.3]{Magnitude_Comparator_WaveDe.png}
	\caption{Simulación del comparador de magnitud con el visor de formas de onda de ModelSim (Base decimal). \label{fig:magnitude_comparator_WaveDe}}
\end{figure}






	\newpage \section{Flip-Flop D \label{sec:s3}}

\begin{center}
	\begin{minipage}{12cm}
		\begin{tcolorbox}[title=Actividad 3]
			 Un flip-flop D puede construirse a partir de dos latch tipo D en cascada (configuración MASTER-SLAVE) de acuerdo al siguiente diagrama:\enter
			 
			 \includegraphics[scale=0.3,center]{FlipFlopD.png}
			 Usar dos instancias del latch D del inciso 2. Repetir los pasos del inciso 1.
		\end{tcolorbox}	
	\end{minipage}
\end{center}

La visualización RTL del flip-flop D, implementado con la directiva \textit{keep}, se muestra en la \autoref{fig:FlipFlop_D_Keep_RTL} y sin esta directiva en la \autoref{fig:FlipFlop_D_NoKeep_RTL}. Se observa que dentro de estas instancias se tiene la descripción por flujo de datos de bajo nivel, o sea, la conexión entre compuertas lógicas (Ver \autoref{fig:FlipFlop_D_Keep_RTL2} y \autoref{fig:FlipFlop_D_NoKeep_RTL2}). Los módulos se diferencian en que el circuito con \textit{keep} conserva más compuertas lógicas que el que no usa la directiva, además de mantener intactas a las señales internas.

La visualización con el \textit{Technology Map Viewer} del flip-flop D, implementado con la directiva \textit{keep}, se muestra en la \autoref{fig:FlipFlop_D_Keep_TMV} y sin esta directiva en la \autoref{fig:FlipFlop_D_NoKeep_TMV}. En la \autoref{fig:FlipFlop_D_Keep_TMV2} y \autoref{fig:FlipFlop_D_NoKeep_TMV2}, se observa que dentro de las instancias se generan celdas lógicas. Los módulos se diferencian en que el circuito con \textit{keep} implementa una celda lógica por cada señal interna declarada en la instancia, mientras que el circuito sin \textit{keep} emplea una sola celda lógica para todo el circuito instanciado. Nótese que, la señal que conecta un latch D con el otro, también genera una celda lógica para el circuito con \textit{keep}, ya que es una señal interna del módulo.

Las simulaciones sin retardos se visualizan en la \autoref{fig:FlipFlop_D_Keep_Wave} para el flip-flop D con directiva \textit{keep} y en la \autoref{fig:FlipFlop_D_NoKeep_Wave} para el flip-flop D sin directiva. Ambos circuitos funcionan de la misma manera, de tal forma que 

\begin{itemize}
	\item \textbf{Si D = 0}: La salida adquiere un valor bajo.
	\item \textbf{Si D = 1}: La salida adquiere un valor alto.
\end{itemize}

Nótese que los cambios en la salida ocurren unicamente cuando CLK tiene un flanco de subida.

Las simulaciones con retardos (modo lento a 85°C) se visualizan en la \autoref{fig:FlipFlop_D_Keep_Wave85} para el flip-flop D con directiva \textit{keep} y en la \autoref{fig:FlipFlop_D_NoKeep_Wave85} para el flip-flop D sin directiva. Se observa que, debido a que se simuló en el peor de los escenarios, la señal de salida Q, tarda en actualizar su valor en ambas simulaciones, no obstante, para el caso del circuito con \textit{keep}, el retardo es ligeramente mayor en comparación con el circuito sin \textit{keep}.

En los Anexos se localiza la descripción del flip-flop D. Tomando como instancia al latch D, declarado en la actividad 2, se realizó la descripción del módulo de forma estructural.

\begin{figure}[ht]
	\centering
	\includegraphics[scale=0.39]{FlipFlop_D_Keep_RTL.png}
	\caption{Diagrama RTL del flip-flop D, descrito con la directiva \textit{keep}. \label{fig:FlipFlop_D_Keep_RTL}}
\end{figure}

\begin{figure}[ht]
	\centering
	\includegraphics[scale=0.34]{FlipFlop_D_NoKeep_RTL.png}
	\caption{Diagrama RTL del flip-flop D, descrito sin la directiva \textit{keep}. \label{fig:FlipFlop_D_NoKeep_RTL}}
\end{figure}

\begin{figure}[ht]
	\centering
	\includegraphics[scale=0.36]{FlipFlop_D_Keep_RTL2.png}
	\caption{Diagrama RTL del flip-flop D, descrito con la directiva \textit{keep} (acercamiento al interior de la instancia). \label{fig:FlipFlop_D_Keep_RTL2}}
\end{figure}

\begin{figure}[ht]
	\centering
	\includegraphics[scale=0.37]{FlipFlop_D_NoKeep_RTL2.png}
	\caption{Diagrama RTL del flip-flop D, descrito sin la directiva \textit{keep} (acercamiento al interior de la instancia). \label{fig:FlipFlop_D_NoKeep_RTL2}}
\end{figure}

\begin{figure}[ht]
	\centering
	\includegraphics[scale=0.38]{FlipFlop_D_Keep_TMV.png}
	\caption{Flip-Flop D (descrito con la directiva \textit{keep}) visto desde el \textit{Technology Map Viewer}. \label{fig:FlipFlop_D_Keep_TMV}}
\end{figure}

\begin{figure}[ht]
	\centering
	\includegraphics[scale=0.33]{FlipFlop_D_NoKeep_TMV.png}
	\caption{Flip-Flop D (descrito sin la directiva \textit{keep}) visto desde el \textit{Technology Map Viewer}. \label{fig:FlipFlop_D_NoKeep_TMV}}
\end{figure}

\begin{figure}[ht]
	\centering
	\includegraphics[scale=0.36]{FlipFlop_D_Keep_TMV2.png}
	\caption{Flip-Flop D (descrito con la directiva \textit{keep}) visto desde el \textit{Technology Map Viewer} (acercamiento al interior de la instancia). \label{fig:FlipFlop_D_Keep_TMV2}}
\end{figure}

\begin{figure}[ht]
	\centering
	\includegraphics[scale=0.33]{FlipFlop_D_NoKeep_TMV2.png}
	\caption{Flip-Flop D (descrito sin la directiva \textit{keep}) visto desde el \textit{Technology Map Viewer} (acercamiento al interior de la instancia). \label{fig:FlipFlop_D_NoKeep_TMV2}}
\end{figure}

\begin{figure}[ht]
	\centering
	\includegraphics[scale=1.3]{FlipFlop_D_Keep_Wave.png}
	\caption{Simulación sin retardos del flip-flop D (descrito con la directiva \textit{keep}) en el visor de formas de onda de ModelSim. \label{fig:FlipFlop_D_Keep_Wave}}
\end{figure}

\begin{figure}[ht]
	\centering
	\includegraphics[scale=1.2]{FlipFlop_D_NoKeep_Wave.png}
	\caption{Simulación sin retardos del flip-flop D (descrito sin la directiva \textit{keep}) en el visor de formas de onda de ModelSim. \label{fig:FlipFlop_D_NoKeep_Wave}}
\end{figure}

\begin{figure}[ht]
	\centering
	\includegraphics[scale=1.3]{FlipFlop_D_Keep_Wave85.png}
	\caption{Simulación con retardos del flip-flop D (descrito con la directiva \textit{keep}) en el visor de formas de onda de ModelSim. \label{fig:FlipFlop_D_Keep_Wave85}}
\end{figure}

\begin{figure}[ht]
	\centering
	\includegraphics[scale=1.2]{FlipFlop_D_NoKeep_Wave85.png}
	\caption{Simulación sin retardos del flip-flop D (descrito sin la directiva \textit{keep}) en el visor de formas de onda de ModelSim. \label{fig:FlipFlop_D_NoKeep_Wave85}}
\end{figure}
	%\newpage \section{Latch D vs Flip-Flop D \label{sec:s4}}

\begin{center}
	\begin{minipage}{12cm}
		\begin{tcolorbox}[title=Actividad 4]
			Repetir el inciso 3 pero indicando codificación \textit{``ONE - HOT''}. ¿Qué códigos \textit{ONE-HOT} fueron utilizados, como los de la tabla normal o la modificada?
		\end{tcolorbox}	
	\end{minipage}
\end{center}

La visualización RTL, del latch D descrito por comportamiento, se muestra en la \autoref{fig:L_D_RTL} y del flip flop D descrito por comportamiento en la \autoref{fig:FF_D_RTL}. Ahora bien, a través del \textit{Chip Planner} se observa que el latch D se implementa en el elemento lógico de la \autoref{fig:L_D_CP} y el flip-flop D en el elemento de la \autoref{fig:FF_D_CP} (Ver en la \autoref{fig:L_D_CP2} y en la \autoref{fig:FF_D_CP2} a los elementos acercados). Aunque pareciera que el latch D y el flip-flop D se efectúan de igual forma en un elemento lógico, se observa en la \autoref{fig:L_D_CP3} que el latch se implementa en la \textit{Look Up Table} de un elemento lógico, mientras que en la \autoref{fig:FF_D_CP3} se ve que el flip-flop se implementa de forma directa en el elemento lógico.

Las simulaciones se visualizan en la \autoref{fig:L_D_Wave} para el latch D y en la \autoref{fig:FF_D_Wave} para el flip-flop D. Se visualiza un correcto funcionamiento de ambos módulos, ya que el latch varía a la salida cuando en CLK hay un cambio al nivel lógico alto, mientras que el flip-flop lo hace con un flaco de subida en CLK.

\begin{figure}[ht]
	\centering
	\includegraphics[scale=0.34]{L_D_RTL.png}
	\caption{Diagrama RTL del latch D, descrito por comportamiento. \label{fig:L_D_RTL}}
\end{figure}

\begin{figure}[ht]
	\centering
	\includegraphics[scale=0.4]{FF_D_RTL.png}
	\caption{Diagrama RTL del flip-flop D, descrito por comportamiento. \label{fig:FF_D_RTL}}
\end{figure}

\begin{figure}[ht]
	\centering
	\includegraphics[scale=0.9]{L_D_CP.png}
	\caption{Latch D visto desde el \textit{Chip Planner} (círculo rojo). \label{fig:L_D_CP}}
\end{figure}

\begin{figure}[ht]
	\centering
	\includegraphics[scale=0.9]{FF_D_CP.png}
	\caption{Flip-Flop D visto desde el \textit{Chip Planner} (círculo rojo). \label{fig:FF_D_CP}}
\end{figure}

\begin{figure}[ht]
	\centering
	\includegraphics[scale=0.6]{L_D_CP2.png}
	\caption{Latch D visto desde el \textit{Chip Planner} (acercamiento). \label{fig:L_D_CP2}}
\end{figure}

\begin{figure}[ht]
	\centering
	\includegraphics[scale=0.6]{FF_D_CP2.png}
	\caption{Flip-Flop D visto desde el \textit{Chip Planner} (acercamiento). \label{fig:FF_D_CP2}}
\end{figure}

\begin{figure}[ht]
	\centering
	\includegraphics[scale=0.4]{L_D_CP3.png}
	\caption{Latch D visto desde el \textit{Chip Planner} (implementación en \textit{Look Up Table} del elemento lógico). \label{fig:L_D_CP3}}
\end{figure}

\begin{figure}[ht]
	\centering
	\includegraphics[scale=0.4]{FF_D_CP3.png}
	\caption{Flip-Flop D visto desde el \textit{Chip Planner} (implementación en elemento lógico). \label{fig:FF_D_CP3}}
\end{figure}

\begin{figure}[ht]
	\centering
	\includegraphics[scale=2]{L_D_Wave.png}
	\caption{Simulación del latch D en el visor de formas de onda de ModelSim. \label{fig:L_D_Wave}}
\end{figure}

\begin{figure}[ht]
	\centering
	\includegraphics[scale=2]{FF_D_Wave.png}
	\caption{Simulación del flip-flop D en el visor de formas de onda de ModelSim. \label{fig:FF_D_Wave}}
\end{figure}

	
	\newpage
	\section{Conclusiones}
	Se logró detectar secuencias de unos y ceros describiendo una máquina de estados finitos de diferentes maneras.
	
	\enter
	
	Se implementó una máquina de estados utilizando las ecuaciones de cada estado para la detección de secuencias y variando el uso de tabla \textit{one hot} y
	\textit{one hot} modificada, analizando los resultados del diagrama RTL.
	
	\enter
	
	Se implementó una máquina de estados finitos por comportamiento, variando el uso del código \textit{minimal bits} y \textit{one hot}, analizando el reporte en su configuración de máquina de estados finitos, comprobando la diferencia que hace el compilador para implementar \textit{minimal bits} o \textit{one hot}.
	
	\enter
	
	Se implementó la detección de secuencia de bit utilizando corrimientos lógicos y aritméticos con 2 y 1 registros, respectivamente. Se encontraron maneras alternativas de resolver el problema de la máquina de estados finitos sin usar tablas de codificación definidas.
	
	\newpage \section{Anexos}

\subsection{Descripciones del hardware}

\lstinputlisting[
	language=verilog,
	label=prog:FSM_OneHot,
	caption={Descripción en Verilog de la máquina de estados finitos, utilizando flip-flops codificados en \textit{One-Hot}.}
]{
	../../Proyectos/FSM_OneHot/FSM_OneHot.v
}

\lstinputlisting[
	language=verilog,
	label=prog:FSM_OneHotM,
	caption={Descripción en Verilog de la máquina de estados finitos, utilizando flip-flops codificados en \textit{One-Hot} modificado.}
]{
	../../Proyectos/FSM_OneHotM/FSM_OneHotM.v
}

\lstinputlisting[
	language=verilog,
	label=prog:FSM_Behavior_MB,
	caption={Descripción en Verilog de la máquina de estados finitos, utilizando codificación \textit{Minimal Bits}.}
]{
	../../Proyectos/FSM_Behavior_MB/FSM_Behavior_MB.v
}

\lstinputlisting[
	language=verilog,
	label=prog:FSM_Behavior_OneHot,
	caption={Descripción en Verilog de la máquina de estados finitos, utilizando codificación \textit{One-Hot}.}
]{
	../../Proyectos/FSM_Behavior_OneHot/FSM_Behavior_OneHot.v
}

\lstinputlisting[
	language=verilog,
	label=prog:FSM_2ShiftRegister,
	caption={Descripción en Verilog de la máquina de estados finitos, utilizando 2 registro de corrimiento.}
]{
	../../Proyectos/FSM_2ShiftRegister/FSM_2ShiftRegister.v
}

\lstinputlisting[
	language=verilog,
	label=prog:FSM_1ShiftRegister,
	caption={Descripción en Verilog de la máquina de estados finitos, utilizando 1 registro de corrimiento.}
]{
	../../Proyectos/FSM_1ShiftRegister/FSM_1ShiftRegister.v
}

\newpage
\subsection{Bancos de pruebas (\textit{Test Benches})}

\lstinputlisting[
language=verilog,
%label=prog:asign,
caption={Banco de prueba para el \autoref{prog:FSM_OneHot}, \autoref{prog:FSM_OneHotM}, \autoref{prog:FSM_Behavior_MB}, \autoref{prog:FSM_Behavior_OneHot}, \autoref{prog:FSM_2ShiftRegister} y \autoref{prog:FSM_1ShiftRegister}.}
]{
	../../Proyectos/FSM_OneHot/simulation/modelsim/FSM_OneHot.vt
}
\end{document}


