\section{Análisis del flip-flop D inferido \label{sec:s4}}



\begin{center}
	\begin{minipage}{10cm}
		En dos proyectos separados describir por comportamiento en uno un latch D simple y en el
		otro un flip-flop D simple (sólo dos entradas: D y clk). Compilar y usar el visor que permita
		observar el elemento lógico donde se implementan el latch y el flip-flop. ¿En qué parte del
		elemento lógico se implementa el latch y en que parte se implementa el flip-flop?
	\end{minipage}
\end{center}

\enter

Al observar el \textit{chip planner}, es posible darse cuenta que, al describir el flip-flop D por comportamiento, se coloca el circuto en el flip-flop D del mismo elemento lógico, esto se ilusta en la \autoref{fig:ffdcomp}.
\begin{figure}[ht]
	\centering
	
	\includegraphics[scale=0.4]{ffd_comportamental.png}
	
	\caption{
		\textit{Chip planner} del Flip-flop D, descrito por comportamiento.
		\label{fig:ffdcomp}
	}
\end{figure}

\enter

Es una diferencia con lo mostrado en la \autoref{fig:ffd}, donde se asignan las \textit{look up tables} debido a la manera en que fue descrito.