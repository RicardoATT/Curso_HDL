\section{Conclusiones}
En conclusión, se implementó la máquina de estados finitos con diferentes métodos de descripción y de codificación.

Para la FSM con flip-flops codificados, se utilizaron los estilos \textit{One-Hot} original y modificado, observándose que son muy pocas las diferencias en la implementación de hardware y en las ecuaciones, no obstante, el funcionamiento resultó ser el mismo.

Para la FSM descrita por comportamiento, se utilizaron los estilos \textit{Minimal Bits} y \textit{One-Hot} para generar los registros de estados correspondientes, sin embargo, al utilizar el \textit{State Machine Viewer} se generó el diagrama de estados, la tabla de transiciones y la tabla de códigos, estando esta ultima en formato \textit{One-Hot} modificado, puesto que el software de Quartus siempre implementa este tipo de codificación para las máquinas de estados finitos.

Para la FSM utilizando registros de corrimientos, se comprobó que, al menos para detectar una secuencia de ceros o unos, se puede utilizar un método diferente, como lo son el uso de registros de corrimiento. Tanto el circuito con dos registros y con un solo registro, funcionaron de manera correcta, observándose que incluso implementaciones complicadas se pueden realizar con hardware simple.

Finalmente, se asignaron los pines correspondientes en la placa de desarrollo para programar el dispositivo y realizar las pruebas pertinentes en hardware.

En los Anexos se pueden encontrar los códigos implementados junto con sus respectivos bancos de pruebas.