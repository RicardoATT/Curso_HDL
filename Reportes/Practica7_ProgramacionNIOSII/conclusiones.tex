\section{Conclusiones}
En conclusión, se implementó el programa de asignación de valores en la salida del procesador NIOS II de manera exitosa.

Utilizando el procedimiento de la presentación vista en clase para programar un NIOS II, se creó un programa que imprime en la pantalla un mensaje y además asigna un valor de 8 bits a la salida, con un cierto retardo. Para ello se empleó la herramienta de \textit{Eclipse} y se usó el lenguaje C para indicarle al procesador la función que debía realizar. 

Se asignaron los pines correspondientes en la placa de desarrollo y se programo en un dispositivo Cyclone V, observándose en los leds asignados, el funcionamiento del programa generado.

En los Anexos se pueden encontrar los códigos implementados.