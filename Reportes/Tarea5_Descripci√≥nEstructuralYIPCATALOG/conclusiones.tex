\section{Conclusiones}
En conclusión, se implementó el circuito en VHDL y en Verilog de forma correcta.

Se comprendió como funcionan las iteraciones For-Loop para generar circuitos, utilizando descripción por comportamiento. De igual forma se entendió la importancia de estos ciclos iterativos para evitar repetir código y hacerlo más legible.

Se investigó en que manera se implementa el For-Loop en Verilog y se diferenció su sintaxis con la de VHDL.

Se diferenciaron los resultados de inicializar o no, las variables que se usan en operaciones aritméticas.

Se implementó la descripción del producto punto de dos vectores empleando el For-Loop y se observaron con el visor RTL a los circuitos instanciados por el ciclo iterativo, y por medio de las simulaciones de forma de onda en ModelSim se visualizó la correcta operación del módulo.

En los Anexos se pueden encontrar los códigos implementados junto con sus respectivos bancos de pruebas. 