\section{Descripción de circuito con diferentes operadores de asignación \label{sec:s1}}

\begin{center}
	\begin{minipage}{12cm}
		\begin{tcolorbox}[title=Actividad 1]
			Capturar el código de la lámina 7 de la presentación de clase (Asignación en verilog (6)). Compilar y observar el resultado de la síntesis con el visor RTL.
		\end{tcolorbox}	
	\end{minipage}
\end{center}

La visualización RTL del circuito con múltiples asignaciones, descrito en Verilog, se muestra en la \autoref{fig:a_circuit1_rtl}. La implementación se hace empleando las compuertas lógicas descritas en el código, no obstante, como se utilizaron operadores de asignación bloqueante y no bloqueante, se usan latches en las señales de salida. Las simulaciones se visualizan en la \autoref{fig:a_circuit1_wave}, en donde se muestra que el módulo opera de manera adecuada, siendo que los latches operan con el cambio de nivel lógico de la señal de \textit{Clock}.

En los Anexos se localiza la descripción en Verilog de este multiplexor. En el código se tiene la declaración de entradas y salidas de la

\begin{figure}[ht]
	\centering
{\tiny {\tiny }}	\includegraphics[scale=0.45]{Assignment_Circuit1_RTL.png}
	\caption{Diagrama RTL del circuito con múltiples asignaciones, descrito en Verilog. \label{fig:a_circuit1_rtl}}
\end{figure}

\begin{figure}[ht]
	\centering
	\includegraphics[scale=2]{Assignment_Circuit1_Wave.png}
	\caption{Simulación del circuito con múltiples asignaciones, descrito en Verilog, con el visor de formas de onda de ModelSim. \label{fig:a_circuit1__wave}}
\end{figure}