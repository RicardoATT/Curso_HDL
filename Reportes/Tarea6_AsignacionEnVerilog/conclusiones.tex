\section{Conclusiones}
En conclusión, se implementaron los tres circuitos de manera adecuada.

Se comprendió como se utilizan los Latches para las asignaciones no inmediatas en un circuito sencillo.

Se diferenciaron a las señales asignadas con operadores bloqueantes y no bloqueantes, gracias a que se visualizó con el visor RTL la manera en que se conectan y operan.

Se comprendió la diferencia de usar un Latch y un Flip-Flop, siendo que el primero opera con niveles lógicos, mientras que el segundo lo hace con flancos de subida o de bajada.

Se entendió un uso interesante de los Latches para habilitar o deshabilitar señales de salida de un circuito.

En todos los casos, se observó con el visor RTL la manera en que se conectan los circuitos, y por medio de las simulaciones de forma de onda en ModelSim, se visualizó la correcta operación de cada dispositivo.

En los Anexos se pueden encontrar los códigos implementados junto con sus respectivos bancos de pruebas. 