\section{Conclusiones}
En conclusión, se implementaron los arreglos de compuertas lógicas, en VHDL y Verilog, de manera correcta.

Se comprendió como se utiliza al bloque \textit{for-generate} para iterar hardware por instanciación (descripción estructural), en comparación con la estructura \textit{for-loop} (vista anteriormente), que se utiliza para la descripción por comportamiento.

Se indagó sobre el uso del bloque \textit{for-generate} en Verilog, considerando aspectos importantes como la declaración de una variable de generación y su importancia ante la redundancia y la legibilidad del código.

Se comprendió la sintaxis de la estructura \textit{for-generate}, tanto en Verilog como en VHDL y se observaron algunas de sus diferencias (la declaración de la variable de generación en Verilog, por ejemplo).

En ambos casos se implementaron los arreglos de compuertas y se observó con el visor RTL a los circuitos instanciados dentro del módulo principal, y por medio de las simulaciones de forma de onda en ModelSim, se visualizó la correcta operación de los dispositivos.

En los Anexos se pueden encontrar los códigos implementados junto con sus respectivos bancos de pruebas. 