\section{\textit{For - Generate} en Verilog \label{sec:s2}}

\begin{center}
	\begin{minipage}{12cm}
		\begin{tcolorbox}[title=Actividad 2]
			Investigar cómo se usa la estructura “\textit{generate}” en verilog.
		\end{tcolorbox}	
	\end{minipage}
\end{center}

\subsection{Definición}
La declaración \textit{generate}, en Verilog, es una construcción muy útil que genera código sintetizable durante el tiempo de elaboración de forma dinámica. El simulador proporciona un código elaborado del bloque \textit{generate} que tiene las siguientes características:

\begin{itemize}
	\item Se generan múltiples instancias de módulo, eliminando la repetición de código.
	\item Se crea una instancia condicional de un bloque de código basado en un parámetro Verilog; sin embargo, el parámetro no está permitido en la declaración de generación.
\end{itemize}

Básicamente proporciona control sobre variables, funciones, tareas y declaraciones de creación de instancias. El bloque de generación se escribe dentro de las palabras clave \textit{generate} y \textit{endgenerate}. \cite{vlsiverify}

\subsection{Sintaxis}
Como se observa en el \autoref{prog:forgenerateinverilog}, la estructura \textit{for-generate} tiene los siguiente puntos importantes:

\lstinputlisting[
language=verilog,
label=prog:forgenerateinverilog,
caption={Sintaxis de la estructura \textit{for-generate}, en Verilog.}
]{
	codes/ForGenerateInVerilog.v
}

\begin{itemize}
	\item \textbf{\textit{genvar}}: Se declara una variable de generación ``i'' de tipo genvar. Esta variable solo es válida dentro del bloque generate.
	\item \textbf{Valor\_inicial}: Valor inicial de la variable ``i''.
	\item \textbf{Valor\_final}: Valor final de la variable ``i''.
	\item \textbf{Incremento:} Incremento de la variable ``i'' en cada iteración. Si no se especifica, el incremento es de 1.
	\item \textbf{Código repetitivo:} El código que se desea generar para cada valor de ``i''. Este código puede incluir declaraciones de variables, asignaciones, instanciaciones de módulos, etc. \cite{chipverify}
\end{itemize}

\subsection{Beneficios del \textit{for-generate}}

\begin{itemize}
	\item \textbf{Mejora la legibilidad del código:} El código repetitivo se organiza de manera más clara y concisa, lo que facilita su comprensión y mantenimiento.
	\item \textbf{Reduce la redundancia:} Se elimina la necesidad de escribir el mismo código múltiples veces, lo que hace que el código sea más compacto y eficiente.
	\item \textbf{Aumenta la flexibilidad:} El código generado puede ser parametrizado, lo que permite adaptar el diseño a diferentes necesidades.
\end{itemize}

\subsection{Consideraciones adicionales}
\begin{itemize}
	\item El \textit{for-generate} solo se puede utilizar para generar código dentro de un bloque \textit{generate}.
	\item La variable de generación ``i'' no se puede utilizar fuera del bloque \textit{for-generate}.
	\item Es importante usar el incremento adecuado para evitar bucles infinitos.
	\item Se debe tener cuidado al utilizar el \textit{for-generate} con código sensible al tiempo, como las señales de reloj. \cite{vlsiverify}
\end{itemize}